%-----------------------------------------------------------------------------------------------------------------------------------------------%
%	The MIT License (MIT)
%
%	Copyright (c) 2016 Jan Küster
%
%	Permission is hereby granted, free of charge, to any person obtaining a copy
%	of this software and associated documentation files (the "Software"), to deal
%	in the Software without restriction, including without limitation the rights
%	to use, copy, modify, merge, publish, distribute, sublicense, and/or sell
%	copies of the Software, and to permit persons to whom the Software is
%	furnished to do so, subject to the following conditions:
%	
%	THE SOFTWARE IS PROVIDED "AS IS", WITHOUT WARRANTY OF ANY KIND, EXPRESS OR
%	IMPLIED, INCLUDING BUT NOT LIMITED TO THE WARRANTIES OF MERCHANTABILITY,
%	FITNESS FOR A PARTICULAR PURPOSE AND NONINFRINGEMENT. IN NO EVENT SHALL THE
%	AUTHORS OR COPYRIGHT HOLDERS BE LIABLE FOR ANY CLAIM, DAMAGES OR OTHER
%	LIABILITY, WHETHER IN AN ACTION OF CONTRACT, TORT OR OTHERWISE, ARISING FROM,
%	OUT OF OR IN CONNECTION WITH THE SOFTWARE OR THE USE OR OTHER DEALINGS IN
%	THE SOFTWARE.
%
%	*************	RESOURCES USED	 ********************
%
%	http://tex.stackexchange.com/questions/5718/package-for-pie-charts
%	http://tex.stackexchange.com/questions/183087/draw-colored-world-us-map-in-latex#183138
%	http://www.texample.net/tikz/examples/simple-flow-chart/
%	http://vizualize.me/#
%	http://devnet.kentico.com/getattachment/fd511a92-e164-40f5-ba51-07c228a49fed/Kentico_fortune500_infographics_FINAL.png
%
%-----------------------------------------------------------------------------------------------------------------------------------------------%


%============================================================================%
%
%	DOCUMENT DEFINITION
%
%============================================================================%

%we use article class because we want to fully customize the page
\documentclass[10pt,A4]{article}	


%----------------------------------------------------------------------------------------
%	ENCODING
%----------------------------------------------------------------------------------------

%we use utf8 since we want to build from any machine
\usepackage[utf8]{inputenc}		

%----------------------------------------------------------------------------------------
%	LOGIC
%----------------------------------------------------------------------------------------

\usepackage{xifthen}
\usepackage{calc}

%----------------------------------------------------------------------------------------
%	FONT
%----------------------------------------------------------------------------------------

% some tex-live fonts - choose your own

%\usepackage[defaultsans]{droidsans}
%\usepackage[default]{comfortaa}
%\usepackage{cmbright}
\usepackage[default]{raleway}
%\usepackage{fetamont}
%\usepackage[default]{gillius}
%\usepackage[light,math]{iwona}
%\usepackage[thin]{roboto} 
\usepackage{hyperref}
\hypersetup{
    colorlinks=true,
    linkcolor=blue,
    filecolor=magenta,      
    urlcolor=cyan,
}

% set font default
\renewcommand*\familydefault{\sfdefault} 	
\usepackage[T1]{fontenc}

% more font size definitions
\usepackage{moresize}		

% awesome font
\usepackage{fontawesome5}


%----------------------------------------------------------------------------------------
%	PAGE LAYOUT  DEFINITIONS
%----------------------------------------------------------------------------------------

%debug page outer frames
%\usepackage{showframe}			

%define page styles using geometry
\usepackage[a4paper]{geometry}		

% for example, change the margins to 2 inches all round
\geometry{top=1cm, bottom=1cm, left=1cm, right=1cm} 	

% use customized header
\usepackage{fancyhdr}				
\pagestyle{fancy}

%less space between header and content
\setlength{\headheight}{-5pt}		

% customize header entries
\lhead{}
\rhead{}
\chead{}

%indentation is zero
\setlength{\parindent}{0mm}

%----------------------------------------------------------------------------------------
%	TABLE /ARRAY DEFINITIONS
%---------------------------------------------------------------------------------------- 

%extended aligning of tabular cells
\usepackage{array}

% custom column width
\newcolumntype{x}[1]{%
>{\raggedleft\hspace{0pt}}p{#1}}%


%----------------------------------------------------------------------------------------
%	GRAPHICS DEFINITIONS
%---------------------------------------------------------------------------------------- 

\usepackage{graphicx}
\usepackage{wrapfig}

% for drawing graphics and charts
\usepackage{tikz}
\usetikzlibrary{shapes, backgrounds}

% use to vertically center content
% credits to: http://tex.stackexchange.com/questions/7219/how-to-vertically-center-two-images-next-to-each-other
\newcommand{\vcenteredinclude}[1]{\begingroup
\setbox0=\hbox{\includegraphics{#1}}%
\parbox{\wd0}{\box0}\endgroup}

% use to vertically center content
% credits to: http://tex.stackexchange.com/questions/7219/how-to-vertically-center-two-images-next-to-each-other
\newcommand*{\vcenteredhbox}[1]{\begingroup
\setbox0=\hbox{#1}\parbox{\wd0}{\box0}\endgroup}

%----------------------------------------------------------------------------------------
%	ICON-SET EMBEDDING
%---------------------------------------------------------------------------------------- 

% at this point we simplify our icon-embedding by simply referring to a set of png images.
% if you find a good way of including svg without conflicting with other packages you can
% replace this part
\newcommand{\icon}[2]{\colorbox{thirdcol}{\makebox(#2, #2){\textcolor{titletext}{\csname fa#1\endcsname}}}}	%icon shortcut
\newcommand{\icontext}[3]{ 						%icon with text shortcut
	\vcenteredhbox{\icon{#1}{#2}} \vcenteredhbox{\textcolor{textcol}{#3}}
}

\newcommand{\icox}[4]{\colorbox{#3}{\makebox(#2, #2){\textcolor{#4}{\csname fa#1\endcsname}}}}	%icon shortcut
\newcommand{\iconbox}[5]{ 						%icon with text shortcut
	\vcenteredhbox{\icox{#1}{#2}{#3}{#5}} \vcenteredhbox{\textcolor{#5}{\colorbox{#3}{#4}}}
}

%----------------------------------------------------------------------------------------
%	Color DEFINITIONS
%---------------------------------------------------------------------------------------- 

\usepackage{xcolor}

%defineColors
\definecolor{orange}{RGB}{255,150,0}
\definecolor{lblue}{RGB}{0,178,255}
\definecolor{darkblue}{RGB}{0,80,130}
\definecolor{darkerblue}{RGB}{0,100,160}
\definecolor{lgray}{RGB}{0,120,200}
\definecolor{powderblue}{RGB}{190,220,255}
\definecolor{darkestblue}{RGB}{0,50,80}


%main color
\colorlet{maincol}{orange}

%secondary colors
\colorlet{secondcol}{lblue}
\colorlet{thirdcol}{darkblue}
\colorlet{fourthcol}{darkerblue}
\colorlet{fifthcol}{lgray}
\colorlet{sixthcol}{darkblue}

%background color
\colorlet{bgcol}{powderblue}

%textcolor
\colorlet{textcol}{darkestblue}

%titletextcolor
\colorlet{titletext}{white}

%sectioncolor
\colorlet{sectcol}{white}

%set a background col for whole page
\pagecolor{bgcol}


%----------------------------------------------------------------------------------------
% 	HEADER
%----------------------------------------------------------------------------------------

% remove top header line
\renewcommand{\headrulewidth}{0pt} 

%remove botttom header line
\renewcommand{\footrulewidth}{0pt}	  	

%remove pagenum
\renewcommand{\thepage}{}	

%remove section num		
\renewcommand{\thesection}{}			


%----------------------------------------------------------------------------------------
%
% 	TIKZ GRAPHICS
%
%----------------------------------------------------------------------------------------


% the chart graphics are outsourced into own files

%----------------------------------------------------------------------------------------
% 	PIE CHART
%----------------------------------------------------------------------------------------
%-----------------------------------------------------------------------------------------------------------------------------------------------%
%	The MIT License (MIT)
%
%	Copyright (c) 2016 Jan Küster
%
%	Permission is hereby granted, free of charge, to any person obtaining a copy
%	of this software and associated documentation files (the "Software"), to deal
%	in the Software without restriction, including without limitation the rights
%	to use, copy, modify, merge, publish, distribute, sublicense, and/or sell
%	copies of the Software, and to permit persons to whom the Software is
%	furnished to do so, subject to the following conditions:
%	
%	THE SOFTWARE IS PROVIDED "AS IS", WITHOUT WARRANTY OF ANY KIND, EXPRESS OR
%	IMPLIED, INCLUDING BUT NOT LIMITED TO THE WARRANTIES OF MERCHANTABILITY,
%	FITNESS FOR A PARTICULAR PURPOSE AND NONINFRINGEMENT. IN NO EVENT SHALL THE
%	AUTHORS OR COPYRIGHT HOLDERS BE LIABLE FOR ANY CLAIM, DAMAGES OR OTHER
%	LIABILITY, WHETHER IN AN ACTION OF CONTRACT, TORT OR OTHERWISE, ARISING FROM,
%	OUT OF OR IN CONNECTION WITH THE SOFTWARE OR THE USE OR OTHER DEALINGS IN
%	THE SOFTWARE.
%
%-----------------------------------------------------------------------------------------------------------------------------------------------%

%counters for chart loop
\newcounter{a}
\newcounter{b}
\newcounter{c}

% draw a slice for a chart
% param 1: Circle form - 90 = quarter, 180 = half, 360 = full
% param 2: scale default=1 (scales only chart, not label text)
% param 3: border color
% param 4: label text color
% param 5: label bg color
% param 6:
\newenvironment{piechart}[5] {

	% draw a slice for a chart
	% param 1: value x of 100
	% param 2: label text
	% param 3: fill color
	% param 4:
	% param 5:
	% param 6:
	\newcommand{\slice}[3] {

		\setcounter{a}{\value{b}}
		\addtocounter{b}{##1}

		%set from angle point
		\pgfmathparse{\thea/100*#1}
	  	\let\pointa\pgfmathresult

		%set toanglepoint
		\pgfmathparse{\theb/100*#1}
	  	\let\pointb\pgfmathresult

		%set midangle
	 	\pgfmathparse{0.5*\pointa+0.5*\pointb}
	  	\let\midangle\pgfmathresult
		
		% draw the slice
	  	\filldraw[fill=##3!100,draw=#3!100, line width=2pt ] (0,0) -- (\pointa:#2) arc (\pointa:\pointb:#2) -- cycle;

	  	% draw label
	  	\node[label=\midangle:\colorbox{#5}{\textcolor{#4}{##2}}] at (\midangle:#2) {};

		\filldraw[fill=#3,draw=none] (0,0) circle (#2/2);
	}

	% execute commands
	\setcounter{a}{0}
	\setcounter{b}{0}
	\begin{tikzpicture}
}
{\end{tikzpicture}}

%----------------------------------------------------------------------------------------
% 	BAR CHART
%----------------------------------------------------------------------------------------
%-----------------------------------------------------------------------------------------------------------------------------------------------%
%	The MIT License (MIT)
%
%	Copyright (c) 2016 Jan Küster
%
%	Permission is hereby granted, free of charge, to any person obtaining a copy
%	of this software and associated documentation files (the "Software"), to deal
%	in the Software without restriction, including without limitation the rights
%	to use, copy, modify, merge, publish, distribute, sublicense, and/or sell
%	copies of the Software, and to permit persons to whom the Software is
%	furnished to do so, subject to the following conditions:
%	
%	THE SOFTWARE IS PROVIDED "AS IS", WITHOUT WARRANTY OF ANY KIND, EXPRESS OR
%	IMPLIED, INCLUDING BUT NOT LIMITED TO THE WARRANTIES OF MERCHANTABILITY,
%	FITNESS FOR A PARTICULAR PURPOSE AND NONINFRINGEMENT. IN NO EVENT SHALL THE
%	AUTHORS OR COPYRIGHT HOLDERS BE LIABLE FOR ANY CLAIM, DAMAGES OR OTHER
%	LIABILITY, WHETHER IN AN ACTION OF CONTRACT, TORT OR OTHERWISE, ARISING FROM,
%	OUT OF OR IN CONNECTION WITH THE SOFTWARE OR THE USE OR OTHER DEALINGS IN
%	THE SOFTWARE.
%
%-----------------------------------------------------------------------------------------------------------------------------------------------%
\newcounter{barcount}


% draw a bar chart
% param 1: width
% param 2: height
% param 3: border color
% param 4: label text color
% param 5: label bg color
% param 6: cat 1 color
\newenvironment{barchart}[8]{

	\newcommand{\barwidth}{0.35}
	\newcommand{\barsep}{0.2}

	% param 1: overall percent
	% param 2: label
	% param 3: cat 1 percent
	% param 4: cat 2 percent
	% param 5: cat 3 percent
	\newcommand{\baritem}[5]{

		\pgfmathparse{##3+##4+##5}
		 \let\perc\pgfmathresult

		\pgfmathparse{#2}
		 \let\barsize\pgfmathresult
	
		\pgfmathparse{\barsize*##3/100}
		 \let\barone\pgfmathresult
	
		\pgfmathparse{\barsize*##4/100}
		 \let\bartwo\pgfmathresult
	
		\pgfmathparse{\barsize*##5/100}
		 \let\barthree\pgfmathresult

		\pgfmathparse{(\barwidth*\thebarcount)+(\barsep*\thebarcount)}
		 \let\barx\pgfmathresult

		\filldraw[fill=#6, draw=none] (0,-\barx) rectangle (\barone,-\barx-\barwidth);
		\filldraw[fill=#7, draw=none] (\barone, -\barx) rectangle (\barone+\bartwo,-\barx-\barwidth);
		\filldraw[fill=#8, draw=none] (\barone+\bartwo,-\barx ) rectangle (\barone+\bartwo+\barthree,-\barx-\barwidth);

		\node [label=180:\colorbox{#5}{\textcolor{#4}{##2}}] at (0,-\barx-0.175) {};
		\addtocounter{barcount}{1}
	}
	\begin{tikzpicture}
	\setcounter{barcount}{0}
	
}
{\end{tikzpicture}}


%----------------------------------------------------------------------------------------
% 	BUBBLE CHART
%----------------------------------------------------------------------------------------
%-----------------------------------------------------------------------------------------------------------------------------------------------%
%	The MIT License (MIT)
%
%	Copyright (c) 2016 Jan Küster
%
%	Permission is hereby granted, free of charge, to any person obtaining a copy
%	of this software and associated documentation files (the "Software"), to deal
%	in the Software without restriction, including without limitation the rights
%	to use, copy, modify, merge, publish, distribute, sublicense, and/or sell
%	copies of the Software, and to permit persons to whom the Software is
%	furnished to do so, subject to the following conditions:
%	
%	THE SOFTWARE IS PROVIDED "AS IS", WITHOUT WARRANTY OF ANY KIND, EXPRESS OR
%	IMPLIED, INCLUDING BUT NOT LIMITED TO THE WARRANTIES OF MERCHANTABILITY,
%	FITNESS FOR A PARTICULAR PURPOSE AND NONINFRINGEMENT. IN NO EVENT SHALL THE
%	AUTHORS OR COPYRIGHT HOLDERS BE LIABLE FOR ANY CLAIM, DAMAGES OR OTHER
%	LIABILITY, WHETHER IN AN ACTION OF CONTRACT, TORT OR OTHERWISE, ARISING FROM,
%	OUT OF OR IN CONNECTION WITH THE SOFTWARE OR THE USE OR OTHER DEALINGS IN
%	THE SOFTWARE.
%
%-----------------------------------------------------------------------------------------------------------------------------------------------%


\newcommand{\bubble}[5]{
	\definecolor{tmpcol}{RGB}{50,50,#5}
	% slice
  	\filldraw[fill=thirdcol,draw=none] (#1,0.5) circle (#3);

  	% outer label
  	\node[label=\textcolor{textcol}{#4}] at (#1,0.7) {};
}

\newcommand{\bubbles}[2]{
	%reset counters
	\setcounter{a}{0}
	\setcounter{c}{150}
	\begin{tikzpicture}[scale=3]
	\foreach \p/\t in {#1} {
	    	\addtocounter{a}{1}
	    	\bubble{\thea/2}{\theb}{\p/25}{\t}{1\p0}
	}
	\end{tikzpicture}
}


%----------------------------------------------------------------------------------------
% 	SQUARE CHART
%----------------------------------------------------------------------------------------
%-----------------------------------------------------------------------------------------------------------------------------------------------%
%	The MIT License (MIT)
%
%	Copyright (c) 2016 Jan Küster
%
%	Permission is hereby granted, free of charge, to any person obtaining a copy
%	of this software and associated documentation files (the "Software"), to deal
%	in the Software without restriction, including without limitation the rights
%	to use, copy, modify, merge, publish, distribute, sublicense, and/or sell
%	copies of the Software, and to permit persons to whom the Software is
%	furnished to do so, subject to the following conditions:
%	
%	THE SOFTWARE IS PROVIDED "AS IS", WITHOUT WARRANTY OF ANY KIND, EXPRESS OR
%	IMPLIED, INCLUDING BUT NOT LIMITED TO THE WARRANTIES OF MERCHANTABILITY,
%	FITNESS FOR A PARTICULAR PURPOSE AND NONINFRINGEMENT. IN NO EVENT SHALL THE
%	AUTHORS OR COPYRIGHT HOLDERS BE LIABLE FOR ANY CLAIM, DAMAGES OR OTHER
%	LIABILITY, WHETHER IN AN ACTION OF CONTRACT, TORT OR OTHERWISE, ARISING FROM,
%	OUT OF OR IN CONNECTION WITH THE SOFTWARE OR THE USE OR OTHER DEALINGS IN
%	THE SOFTWARE.
%
%-----------------------------------------------------------------------------------------------------------------------------------------------%

\newcommand{\squares}[2]{
	%reset counters
	\setcounter{a}{0}
	\setcounter{b}{0}
	\setcounter{c}{50}
	\begin{tikzpicture}[scale=3]
	\foreach \p/\t in {#1} {
		\setcounter{a}{\value{b}}
	    	\addtocounter{b}{\p}
	    	\square{\thea/100*#2}{\theb/100*#2}{\p\%}{\t}{thirdcol}
	}
	\end{tikzpicture}
}

\newcommand{\square}[5] {
 	\pgfmathparse{#1+0.5*(#2-#1)}
  	\let\midangle\pgfmathresult

	\draw[draw=sectcol] (0.4, \midangle) -- (0.6,\midangle);
	% slice
  	\filldraw[fill=#5!100,draw=bgcol!100, line width=3pt] (0,#1) -- (0.5,#1) -- (0.5,#2) -- (0,#2) -- cycle;
  	% outer label
  	\node[label=360:\colorbox{sectcol}{\textcolor{textcol}{#4}}] at (0.55,\midangle) {};
}



%----------------------------------------------------------------------------------------
% 	TIMELINE CHART
%----------------------------------------------------------------------------------------
%-----------------------------------------------------------------------------------------------------------------------------------------------%
%	The MIT License (MIT)
%
%	Copyright (c) 2016 Jan Küster
%
%	Permission is hereby granted, free of charge, to any person obtaining a copy
%	of this software and associated documentation files (the "Software"), to deal
%	in the Software without restriction, including without limitation the rights
%	to use, copy, modify, merge, publish, distribute, sublicense, and/or sell
%	copies of the Software, and to permit persons to whom the Software is
%	furnished to do so, subject to the following conditions:
%	
%	THE SOFTWARE IS PROVIDED "AS IS", WITHOUT WARRANTY OF ANY KIND, EXPRESS OR
%	IMPLIED, INCLUDING BUT NOT LIMITED TO THE WARRANTIES OF MERCHANTABILITY,
%	FITNESS FOR A PARTICULAR PURPOSE AND NONINFRINGEMENT. IN NO EVENT SHALL THE
%	AUTHORS OR COPYRIGHT HOLDERS BE LIABLE FOR ANY CLAIM, DAMAGES OR OTHER
%	LIABILITY, WHETHER IN AN ACTION OF CONTRACT, TORT OR OTHERWISE, ARISING FROM,
%	OUT OF OR IN CONNECTION WITH THE SOFTWARE OR THE USE OR OTHER DEALINGS IN
%	THE SOFTWARE.
%
%-----------------------------------------------------------------------------------------------------------------------------------------------%

% define global counters
\newcounter{yearcount}


\newcounter{leftcount}

% env cvtimeline
%
% creates a vertical cv timeline
%
% param 1: start year
% param 2: end year
% param 3: overall width
% param 4: overall height
\newenvironment{cvtimeline}[4]{
	
	\newcommand{\cvcategory}[2]{
		\node[label=\mbox{\colorbox{##1}{\strut\hspace{2pt}}\colorbox{white}{\textcolor{textcol}{##2}}}] at (0,-5) {}; %start year
	}

	\newcommand{\bxwidth}{4.5}
	\newcommand{\bxheight}{2}


	% creates a stretched box as cv entry headline followed by two paragraphs about 
	% the work you did
	% param 1:	event start month/year
	% param 2:	event end month/year
	% param 3:  event name
	% param 4:	institution (where did you work / study)
	% param 5:	what was your position
	% param 6:	color
	% param 7:  level (position, use minus for left placement)
	\newcommand{\cvevent}[7] {
		
		\foreach \monthf/\yearf in {##1} {
			\foreach \montht/\yeart in {##2} {

				\pgfmathparse{#3/\fullrange*((\yearf-#1)+(\monthf/12))}
  				\let\startexp\pgfmathresult
				\pgfmathparse{#3/\fullrange*((\yeart-#1)+(\montht/12))}
  				\let\endexp\pgfmathresult
				\pgfmathparse{1/(\endexp-\startexp+1)}
  				\let\lenexp\pgfmathresult
				\pgfmathparse{0.5*\endexp+0.5*\startexp}
  				\let\midexp\pgfmathresult

				%\filldraw[fill=##6,draw=none, opacity=0.9] (-0.15-##7,\startexp) rectangle (-0.15-##7-0.5,\endexp);
				\draw[draw=##6, line width=1.5pt] (0, \startexp) -- (1,\startexp);

				\node[label={[label distance=0]0:\colorbox{##6}{\strut}\colorbox{white}{\textcolor{gray}{##1}\hspace{3pt}\textcolor{textcol}{##3}}}] at (0.5, \startexp) {};
			}
			\addtocounter{leftcount}{1}
		}
	}

	%--------------------------------------------------------------------------------------
	%	BEGIN
	%--------------------------------------------------------------------------------------

	\begin{tikzpicture}

	\setcounter{leftcount}{1}

	%calc fullrange= number of years
 	\pgfmathparse{(#2-#1)}
  	\let\fullrange\pgfmathresult
	\draw[draw=textcol,line width=4pt] (0,0) -- (0,#3) ;	%the timeline

	%for each year put a horizontal line in place
	\setcounter{yearcount}{1}
	\whiledo{\value{yearcount} < \fullrange}{
		\draw[fill=white,draw=textcol, line width=2pt]  (0,#3/\fullrange*\value{yearcount}) circle (0.1);
		\stepcounter{yearcount}
	}

	%start year
	\filldraw[fill=white!100,draw=textcol,line width=3pt] (0,-0.5) circle (0.5);
	\node[label=\textcolor{textcol}{\textbf{\small#1}}] at (0,-0.85) {}; 

	%end year
	\filldraw[fill=white!100,draw=textcol,line width=5pt] (0,#3+0.75) circle (0.75);
	\node[label=\textcolor{textcol}{\textbf{\large#2}}] at (0,#3+0.42) {}; 


	
}%end begin part of newenv
{\end{tikzpicture}}

%----------------------------------------------------------------------------------------
% 	FACT BUBBLE
%----------------------------------------------------------------------------------------
%-----------------------------------------------------------------------------------------------------------------------------------------------%
%	The MIT License (MIT)
%
%	Copyright (c) 2016 Jan Küster
%
%	Permission is hereby granted, free of charge, to any person obtaining a copy
%	of this software and associated documentation files (the "Software"), to deal
%	in the Software without restriction, including without limitation the rights
%	to use, copy, modify, merge, publish, distribute, sublicense, and/or sell
%	copies of the Software, and to permit persons to whom the Software is
%	furnished to do so, subject to the following conditions:
%	
%	THE SOFTWARE IS PROVIDED "AS IS", WITHOUT WARRANTY OF ANY KIND, EXPRESS OR
%	IMPLIED, INCLUDING BUT NOT LIMITED TO THE WARRANTIES OF MERCHANTABILITY,
%	FITNESS FOR A PARTICULAR PURPOSE AND NONINFRINGEMENT. IN NO EVENT SHALL THE
%	AUTHORS OR COPYRIGHT HOLDERS BE LIABLE FOR ANY CLAIM, DAMAGES OR OTHER
%	LIABILITY, WHETHER IN AN ACTION OF CONTRACT, TORT OR OTHERWISE, ARISING FROM,
%	OUT OF OR IN CONNECTION WITH THE SOFTWARE OR THE USE OR OTHER DEALINGS IN
%	THE SOFTWARE.
%
%-----------------------------------------------------------------------------------------------------------------------------------------------%

% draw a circle with facts
% param 1: fact text
% param 2: scale default=1 (scales only chart, not label text)
% param 3: big border color
% param 4: second border color
% param 5: label bg color
\newcommand{\factbubble}[5]{
	\begin{tikzpicture}
	\pgfmathparse{#2*2}
	\let\pbxwidth\pgfmathresult
		\filldraw[fill=#3,draw=none] (0,0) circle (#2 * 1.5);
		\filldraw[fill=#5,draw=#4, line width=3.5pt] (0,0) circle (#2 * 1.2);
		\node at (0,0) {
			\parbox{\pbxwidth cm}{
				\begin{center}	
					#1
				\end{center}
			}
		};
	\end{tikzpicture}
}


%----------------------------------------------------------------------------------------
%	custom sections
%----------------------------------------------------------------------------------------

% create a coloured box with arrow and title as cv section headline
% param 1: section title
%
\newcommand{\cvsection}[2] {
\textcolor{titletext}{\uppercase{\textbf{#1}}}
}

\newcommand{\cvsect}[4]{
	\textcolor{#3}{\hrule}
	\colorbox{#3}{ {\cvsection{#1}{#4}}}
}

% create a coloured arrow with title as cv meta section section
% param 1: meta section title
%
\newcommand{\metasection}[2] {
	\begin{tabular*}{1\textwidth}{ l l }
		#1&#2\\[12pt]
	\end{tabular*}
}

%----------------------------------------------------------------------------------------
%	 CV EVENT
%----------------------------------------------------------------------------------------

% creates a stretched box as 
\newcommand{\cveventmeta}[2] {
	\mbox{\mystrut \hspace{87pt}\textit{#1}}\\
	#2
}

%----------------------------------------------------------------------------------------
% STRUTS AND RULES
%----------------------------------------- -----------------------------------------------

% custom strut
\newcommand{\mystrut}{\rule[-.3\baselineskip]{0pt}{\baselineskip}}

% colored rule and text for chart legends, wrapped in parbox
% param 1: text
% param 2: width in cm or pt, em ...
% param 3: color
\newcommand{\legend}[3]{\parbox[t]{#2}{\textcolor{#3}{\rule{#2}{4pt}}\\#1}}

%----------------------------------------------------------------------------------------
% CUSTOM LOREM IPSUM
%----------------------------------------------------------------------------------------
\newcommand{\lorem}{Lorem ipsum dolor sit amet, consectetur adipiscing elit. Donec a diam lectus.}


%============================================================================%
%
%
%
%	DOCUMENT CONTENT
%
%
%
%============================================================================%
\begin{document}


%use our custom fancy header definitions
\pagestyle{fancy}	


%----------------------------------------------------------------------------------------
%	TITLE HEADLINE
%----------------------------------------------------------------------------------------
\mystrut
\vspace{-12pt}

\hspace{-16pt}\begin{tabular*}{1\textwidth}{ c c c}
	\parbox[c]{0.43\linewidth}{
		\colorbox{thirdcol}{\HUGE{\textcolor{titletext}{\textbf{\uppercase{Achraf Najmi}}}}}\\
		\Large{\textcolor{thirdcol}{\textsc{Electro Info Physicist}}}\\
	}&
	\parbox{0.22\textwidth}{
	\icontext{BirthdayCake}{12}{Born :  7/24/1994}\\
	\icontext{MapMarker}{12}{Casablanca, Morocco}\\
	\href{callto:+212622902221}{\icontext{Mobile}{12}{+212 6 22 90 22 21}}\\
	\icontext{Car}{12}{Driver's license : B}\\
	}&
	\parbox{0.4\textwidth}{
	\href{mailto:najmi.achraf@gmail.com}{\icontext{At}{12}{najmi.achraf@gmail.com}}\\
	\href{https://www.github.com/DeepEastWind}{\icontext{Github}{12}{github.com/DeepEastWind}}\\
	\href{https://stackoverflow.com/users/12854948}{\icontext{StackOverflow}{12}{stackoverflow.com/users/12854948}}\\
	\href{https://www.linkedin.com/in/achraf-d-najmi}{\icontext{Linkedin}{12}{linkedin.com/in/achraf-d-najmi}}\\
	}
\end{tabular*}

% manage space by reducing font size
\small

\vspace{-5pt}

\hspace{-10pt}\begin{minipage}{0.59\textwidth}

%----------------------------------------------------------------------------------------
%	FACTS
%----------------------------------------------------------------------------------------

	\mbox{
		\parbox[c][3cm][c]{0.29\textwidth}{
		\textcolor{textcol}{I am a graduate in Physics Electronic with working experience in research projects.}
		}
		\hspace{10pt}
		\parbox[c][3cm][c]{0.32\textwidth}{
			\begin{center}
			\factbubble{\huge{\textcolor{titletext}{\textbf{BGS}}}\\\small{\textcolor{titletext}{\textbf{Physics Electronic}}}}{1}{maincol}{titletext}{thirdcol}\\
			\textcolor{textcol}{as latest degree from}\\
			\textcolor{textcol}{\textbf{University Hassan II of Casablanca : Faculty of Science Ben M'sik.
}}
			\end{center}
		}
		\hspace{10pt}
		\parbox[c][3cm][c]{0.29\textwidth}{
		\textcolor{textcol}{I am always looking for exciting new interdisciplinary projects related to physics.}
		}
	}
	\vspace{20pt}

%----------------------------------------------------------------------------------------
%	SKILLS AND TECHNOLOGIES
%----------------------------------------------------------------------------------------

	\cvsect{Skills \& Technologies}{0.49}{thirdcol}{textcol}\\
	\mbox{
		
		% TEXT BOX
		\parbox[b][130pt][c]{0.35\textwidth}{
			\textcolor{textcol}{Most of my contribution to my work is in the field of IT development and maintenance.}
						
			% LANGUAGES
			\cvsect{Languages}{0.49}{thirdcol}{textcol}\\[4pt]
			\iconbox{Language}{12}{fifthcol}{French (B2)}{titletext}\\
			\iconbox{Language}{12}{fifthcol}{English (A2)}{titletext}\\
		}

		% PIE CHART	
		\begin{piechart}{360}{1.75}{bgcol}{textcol}{sectcol}
			\slice{18}{Projects}{fifthcol}
			\slice{29}{Development}{maincol}
			\slice{29}{Maintenance}{thirdcol}
			\slice{14}{Research}{secondcol}
			\slice{10}{Design}{fourthcol}
		\end{piechart}\\
	}
	\begin{center}
	\vspace{-10pt}
	\begin{tikzpicture}
		\draw[draw=titletext,dashed, opacity=0.5] (4,0) -- (-4,0);
	\end{tikzpicture}
	\end{center}

	\vspace{10pt}
	% TEXT BOX
	\parbox[b][30pt][c]{0.73\textwidth}{
		% DEV LANGUAGE
		\cvsect{Dev Language}{0.49}{thirdcol}{textcol}\\[4pt]
		\iconbox{Python}{12}{maincol}{Python 3}{textcol}
		\hspace{15pt}
		\legend{Language}{1.6cm}{maincol} \legend{Front End}{1.5cm}{secondcol} \legend{Back End}{1.5cm}{thirdcol}
	}
	
	% BAR CHART
	\vspace{9pt}\mbox{\hspace{-14pt}
		\begin{barchart}{10}{5.2}{sectcol}{textcol}{sectcol}{maincol}{secondcol}{thirdcol}
% n1*x + n2*x + n3*x... = 100, Ex:{x=0.459770114942529, n1=80} -> x*n1 = 36.7816
			% \baritem{50}{\faPython \hspace{1pt} Python 3}{36.7816}{33.3333}{29.885} 
			\baritem{50}{\faCode \hspace{1pt} Syntax}{100}{0}{0}
			\baritem{50}{\faBoxes \hspace{1pt} Libraries}{0}{0}{88}
			\baritem{50}{\faLaptopCode \hspace{1pt} GUI}{0}{82.5}{0}
			\baritem{50}{\faChartPie \hspace{1pt} Data Analysis}{0}{0}{77}
			
		\end{barchart}
		
		\hspace{10pt}
	
		% TEXTBOX
		\parbox[b][70pt][c]{0.26\textwidth}{
		
		\vspace{5pt}
		\iconbox{Code}{8}{maincol}{Syntax:}{textcol}
		
		\hspace{10pt}
		\iconbox{Python}{8}{maincol}{3.7 \& 3.8}{textcol}
		
		\iconbox{Boxes}{8}{thirdcol}{Libraries:}{titletext}
		
		\hspace{10pt}
		\iconbox{Terminal}{8}{thirdcol}{built-in}{titletext}
		
		\hspace{10pt}
		\iconbox{FileCode}{8}{thirdcol}{pyinstaller}{titletext}
		
		\iconbox{LaptopCode}{8}{secondcol}{GUI:}{textcol}
			
		\hspace{10pt}	
		\iconbox{Feather}{8}{secondcol}{tkinter}{textcol}
		
		\hspace{10pt}
		\iconbox{ChartArea}{8}{secondcol}{matplotlib}{textcol}
		
		\iconbox{ChartPie}{8}{thirdcol}{Data Analysis:}{titletext}
		
		\hspace{10pt}
		\iconbox{Superscript}{8}{thirdcol}{sympy}{titletext}
		
		\hspace{10pt}	 		
		\iconbox{SortNumericDown}{8}{thirdcol}{numpy}{titletext}	
		
		\hspace{10pt}
		\iconbox{Table}{8}{thirdcol}{pandas}{titletext}
		
		}
	}\\
	
	% TEXTBOX
	\parbox[b][35pt][c]{0.73\textwidth}{\textcolor{textcol}{Currently, I am developing realtime applications with Python, and creating a GUI with tkinter, and extracting from the script `file.py` an executable file by pyinstaller using auto py to exe.}}
	\begin{center}
	\begin{tikzpicture}
		\draw[draw=titletext,dashed, opacity=0.5] (4,0) -- (-4,0);
	\end{tikzpicture}
	\end{center}
	\vspace{-10pt}
	\begin{center}
	\mbox{
		\parbox[b][30pt][c]{0.4\textwidth}{
			% OS
			\cvsect{OS}{0.49}{thirdcol}{textcol}\\[4pt]
			\iconbox{Windows}{12}{secondcol}{Windows}{textcol}
			\iconbox{Linux}{12}{secondcol}{Linux}{textcol}
		}
		\parbox[b][30pt][c]{0.6\textwidth}{
			\textcolor{textcol}{I regularly use for my development projects PyCharm Community and Visual Studio as IDE, and I use for schematic capture and simulation Proteus CAD.}
		}
	}\newline 
	
	\mbox{	
		\bubbles{6/PyCharm, 3.9/\faMarkdown \hspace{0.1em} MD, 5.2/Visual Studio, 3.4/\LaTeX, 4.5/Proteus CAD, 3/\faCodeBranch \hspace{0.1em} \faGit}{\cvsection{Utilitaires}}	
	}
		
	\end{center}
	\vspace{-10pt}
	
%---------------------------------------------------------------------------------------
%	ACTIVITIES
%----------------------------------------------------------------------------------------
	\cvsect{Activities}{0.49}{thirdcol}{textcol}\\[10pt]
	\mbox{
		
		% SQUARE BARS
		\parbox[b][3cm][c]{4.6cm}{
			\squares{24/\faBicycle \hspace{0.5pt} Sport,27/\faChess \hspace{1pt} Brain Games,30/\faNewspaper \hspace{1pt} Tech News,33/\faPlayCircle \hspace{1pt} Tutorials}{1}
			}
		
		% FACT BUBBLE
		\parbox[b][3cm][c]{3cm}{
			\begin{center}
			\factbubble{\HUGE{\textcolor{titletext}{\textbf{5}}}\\\small{\textcolor{titletext}{\textbf{public \\ repos}}}}{0.85}{maincol}{titletext}{thirdcol}\\
			\textcolor{textcol}{in my}\\
			\textcolor{textcol}{\textbf{GitHub Account \\ \faGithub}}
			\end{center}
		}

		% TEXT BOX
		\parbox[b][3cm][c]{3.5cm}{
			\textcolor{textcol}{Among the dominant activities that I practice is to browse the internet in search of news on the computer world and trained on all the tutorials that interests me.}
		}
	}

\end{minipage}
%\begin{minipage}{0.05\textwidth}
%	\begin{center}
%		\begin{tikzpicture}
%			\draw[draw=titletext,dashed, opacity=0.5] (0,-12) -- (0,12);
%		\end{tikzpicture}
%	\end{center}
%\end{minipage}
\hspace{2pt}
\begin{minipage}{0.46\textwidth}
%---------------------------------------------------------------------------------------
%	EXPERIENCE / EDUCATION
%----------------------------------------------------------------------------------------
\cvsect{Experience \& Education}{0.49}{thirdcol}{textcol}\\% [16pt]


%----------------------------------------------------------------------------------------
%	Physique Électronique
%----------------------------------------------------------------------------------------
\hspace{5pt}\colorbox{maincol}{\faRobot \hspace{1pt} Electronic ~•~ \faChargingStation \hspace{1pt} Electrical Engineering ~•~ \faLaptop \hspace{1pt} Digital}

\hspace{60pt}\colorbox{maincol}{\faBroadcastTower \hspace{1pt} Telecommunication ~•~ \faSitemap \hspace{1pt} Network}\\

\hspace{60pt}\mbox{\legend{Experience}{1.7cm}{thirdcol} \legend{Education}{1.5cm}{maincol} \legend{Projects}{1.2cm}{fifthcol} \legend{Events}{1cm}{secondcol}}

\vspace{-55pt}
\begin{center}

% TIMELINE fifthcol
\begin{cvtimeline}{2017}{2022}{21}{\linewidth}
% {maincol}  {secondcol}  {thirdcol}

% \cvevent{6/2012}{6/2012}{High School Degree \newline Series : Experimental Sciences \newline Option : Science of Life and Earth}{where}{what}}{maincol}{0}

% \cvevent{9/2012}{9/2012}{Begin University Studies}{where}{what}{maincol}{0}

% \cvevent{6/2013}{3/2021}{Computer Maintenance \newline • Diagnostic and reparation of Hardware \& Software...}{where}{what}{thirdcol}{0.25}

\cvevent{1/2017}{3/2021}{Computer Maintenance \newline • Diagnostic and reparation of Hardware \& Software...}{where}{what}{thirdcol}{0.25}

\cvevent{6/2017}{12/2016}{Associate of General Studies (AGS) \newline • Series : Matter Science Physics}{where}{what}{maincol}{0.15}

\cvevent{9/2017}{6/2019}{Beginning of Bachelor studies}{where}{what}{maincol}{0.15}

\cvevent{2/2018}{6/2018}{Start the Graduation Project \newline • Design of a photovoltaic system and realization of a solar tracker}{where}{what}{fifthcol}{0.05}

\cvevent{6/2018}{6/2018}{Presentation of Project Graduation}{where}{what}{secondcol}{0.05}

\cvevent{6/2019}{6/2019}{Bachelor of General Studies (BGS) \newline • Series : Matter Science Physics \newline • Course : Electronic}{where}{what}{maincol}{0.15}

\cvevent{9/2019}{3/2021}{Beginning of learning Python}{where}{what}{maincol}{0.15}

\cvevent{12/2019}{2/2021}{Start the MathPy Project \newline • Modern Scientific Calculator }{where}{what}{fifthcol}{0.05}

\cvevent{4/2020}{3/2021}{Set MathPy Project in GitHub \newline \href{https://www.github.com/DeepEastWind/MathPy}{\textcolor{textcol}{\faGithub} \hspace{1pt} github.com/DeepEastWind/MathPy \newline (PRIVATE PROJECT)} \newline MathPy Project is constantly updating}{where}{what}{secondcol}{-0.05}

\cvevent{9/2020}{10/2020}{Start the Gmail Project \newline • CSV Google-mail Sender \newline \href{https://github.com/DeepEastWind/Gmail}{\textcolor{textcol}{\faGithub} \hspace{1pt} github.com/DeepEastWind/Gmail}}{where}{what}{fifthcol}{0.05}

\cvevent{2/2021}{3/2021}{Start the Hydrogeology Project \newline • Groundwater Hydrology \newline \href{https://github.com/DeepEastWind/Hydrogeologie}{\textcolor{textcol}{\faGithub} \hspace{1pt} github.com/DeepEastWind \newline /Hydrogeologie (PRIVATE PROJECT)}}{where}{what}{fifthcol}{0.05}

\end{cvtimeline}
\end{center}
\vspace{-40pt}\end{minipage}
%============================================================================%
%
%
%
%	DOCUMENT END
%
%
%
%============================================================================%
\end{document}
